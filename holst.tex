\documentclass[b5paper]{ujreport}
\usepackage{bxpapersize}
\begin{document}

\begin{titlepage}
\begin{center}
\vspace*{3pt}
{\large\itshape SKSAT vol.1}
\vspace{12pt} \\
\begin{tabular}{rl}
著 sksat
\end{tabular}
\vspace{3pt} \\
\today \vspace{12pt} \\
\end{center}
\end{titlepage}

\chapter*{序}

\chapter{数値計算の理論}
\section{はじめに}
現在僕は,ロケットの軌道や,N体問題
\footnote{後述するが,簡単に言うと,「相互作用する3体以上の質点系」のこと.本稿においては複数の天体の運動を考える問題と思ってくれて構わない.}
のコンピュータ・シミュレーションを行っている.

\section{シミュレーションとは}
シミュレーションとはそもそも,何らかのシステムが従っている法則を抽出、モデル化し、模擬することである.
その「模擬」する方法は特に決まっているわけではなく,紙とペンかもしれないし,小さな模型\footnote{風洞実験などがこれに当てはまる.}かもしれないし,コンピュータプログラムかもしれない.

\end{document}

