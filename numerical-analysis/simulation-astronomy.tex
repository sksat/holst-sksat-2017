\chapter{シミュレーション天文学}

\section{シミュレーション天文学とは}
天文学と聞くと,望遠鏡で星を観測して云々,というイメージしか湧かないかもしれないが,最近の天文学というのは,実際に天体観測を行うもの
\footnote{実際に天体観測を行う天文学の中にも,目に見える可視光のほかに,赤外線を使って観測を行う赤外線天文学,X線を使うX線天文学,電波を使う電波天文学,最近では,ニュートリノを使ったニュートリノ天文学というものもある.}
の他に,天体の運動や構造などの理論を考える理論天文学,そして,今回取り上げるような,天体のシミュレーションをするシミュレーション天文学など,様々な分野がある.


特に,シミュレーション天文学は,惑星の形成,銀河の衝突,ブラックホールの降着円盤,銀河の衝突,宇宙の大規模構造など,実際に観測することが難しい,もしくは観測できていても,それを確実に説明する理論がなかったり,計算量が膨大になったりするもののシミュレーションを行う.

国立天文台の4D2Uのウェブサイトなどを見ると,最新のシミュレーション天文学の成果を見ることができる.

\section{N体シミュレーション}
さて,このようにさまざまな天文現象を扱うシミュレーション天文学だが,これらのシミュレーションは,実際にはどのように行っているのだろうか.
ここでは,シミュレーション天文学の一分野であるN体シミュレーションについて考えてみよう.


N体シミュレーションというのは,その名の通り,N個
\footnote{Nは任意の自然数}
の物体の運動をシミュレーションするもので,多体シミュレーションとも呼ばれる.


天文学でN体シミュレーションが用いられる分野としては,惑星形成理論などがある.

実は,現在の太陽系がどのように形成されたかということには謎が多く,様々な形成シナリオが提案されているが,いったいどれが正しいのか,ということは中々検証できるものではない.
\footnote{シミュレーション以外での検証アプローチとして,太陽系形成初期の物質を調べる,というものがある.しかし,地球にある物質は熱変性や風化を受けているため,太陽系形成初期そのままの物質は全く残っていない.そのため,熱変性や風化を受けていない小惑星のサンプルを入手しよう,という試みが,あの「はやぶさ」である.}

そのため,微惑星をコンピューター上で様々な条件で衝突,合体させ,現在の太陽系の形成の条件を探る,というのが惑星形成のシミュレーションであり,これはたくさんの微惑星の運動を計算しなければならないので,N体シミュレーションの一種である.
