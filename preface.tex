\chapter*{序文}
さて,今年もやって参りました,小冊子のお時間です.

地学部の部誌,「ホルスト」における小冊子というものは,「なんか一人で何個か原稿書いちゃったし実質部誌!w」みたいなテンションで生み出される,同じ人間が書いた原稿の集合体です.


ちなみにこの小冊子,「SKSAT」は,「タキスト」,「SKMT」,「SKMT-2」に続く,(たぶん)4つ目の小冊子です.
まあ,「SKMT-2」は去年僕が部誌の編集作業に追われながら書いたもので,クオリティも厚みも無かったのでノーカンかもしれませんが.


去年はちょっと,いやかなり,なんというかショボかったので,今年はずいぶん早くから原稿を書くことを考えていたのですが,おかしいですね,今この文章を書いているのは9/7のことです.
しかも完成していない原稿がありますこれはアカン.

書くネタは考えていても直前になって書き始めたり,突然書きたいネタが増えたり,\LaTeX の学習に時間がかかったりしたのがダメだった.

まあ,Word\sout{とかいうちょっとアレなソフトウェア}で書くよりは圧倒的成長と圧倒的進捗をしたので良かったと思っています.


肝心の中身についてですが,本当は数値計算もっとやりたかったし、JAXAの新しいロケットエンジンのこととか書きたかったんですが,できませんでした.はい.
もしかしたら未練から文化祭後に書いて来年の部誌にcommitしてしまうかもね.今年引退ですけど.


まあこんな感じのテンションで書いているので,温かい目で読んでやってください.


あと,この原稿の\LaTeX ソースコードはGitHubに上げてある
\footnote{https://github.com/sk2sat/holst-sksat}ので,間違い等あればそちらにIssueなりPRなりしていただけるとありがたいです.
